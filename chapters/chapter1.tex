\section{Introduction}

Operational Amplifiers, also known as Op-amps, are basically a voltage amplifying device designed to be used with components like capacitors and resistors, between its in/out terminals. They are essentially a core part of analog devices. Feedback components like these are used to determine the operation of the amplifier. The amplifier can perform many different operations, giving it the name Operational Amplifier.

One key to the usefulness of these little circuits is in the engineering principle of feedback, particularly negative feedback, which constitutes the foundation of almost all automatic control processes. The principles presented in this section, extend well beyond the immediate scope of electronics. It is well worth the electronics student’s time to learn these principles and learn them well.

Operational amplifiers can have either a closed-loop operation or an open-loop operation. The operation (closed-loop or open-loop) is determined by whether or not feedback is used. Without feedback the operational amplifier has an open-loop operation. This open-loop operation is practical only when the operational amplifier is used as a coop- erator (a circuit which compares two input signals or compares an input signal to some fixed level of voltage). As an amplifier, the open-loop operation is not practical because the very high gain of the operational amplifier creates poor stability. (Noise and other unwanted signals are amplified so much in open-loop operation that the operational am- plifier is usually not used in this way.) Therefore, most operational amplifiers are used with feedback (closed-loop operation).