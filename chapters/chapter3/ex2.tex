\subsection{High-Current Voltage Follower}

The voltage follower's low output impedance makes it a good circuit for driving current
into a low-impedance load, but it's important to remember that most op-amps are not
designed to deliver large output currents. The most basic circuit for buffering an op-amp's
output current is the following:

\begin{figure}[H]
    \centering
    \includegraphics[width=0.7\textwidth]{graphics/question/c3_ex2.png}
    \caption{Opamp follower circuit}
    \label{fig:chap3_question_ex2}
\end{figure}

The voltage at the positive pin of the Opamp is copied to $V_{OUT}$. In this schematic, R16 is
used to simulate a load device, which can be a motor or an high power LED. However, in
this case, there is a high current can pass the load.\\
Students are proposed to run the simulation with bias configuration, capture the results
and place them in the report.\\
Finally, your computations go here to explain the results.

\begin{figure}[H]
    \centering
    \includegraphics[width=0.7\textwidth]{graphics/ex2_sim.png}
    \caption{The bias point simulation}
    \label{fig:chap3_solution_ex2}
\end{figure}


We have: $V_{(+)} = V_{(-)} = V_E = 3V$
\[I_E = \frac{V_E}{R16} = \frac{3V}{1k\Omega} = 3mA\]
\[I_C = \frac{\beta}{\beta + 1} I_E = \frac{100}{101} \cdot 3mA = 2.97mA\]
\[I_B = I_E - I_C = 3mA - 2.97mA = 0.03mA\]
\[V_B = V_E + 0.7V = 3V + 0.7V = 3.7V\]